\documentclass{article}
\usepackage{graphicx}
\usepackage{listings}
\usepackage{hyperref}

\title{CS 3502: Multi-Threaded Programming and IPC Project}
\author{Aly Traore \ CS 3502/W02 \ NETID: Atraor13@students.kennesaw.edu}
\date{March 2025}

\begin{document}

\maketitle

\section{Introduction}
This project demonstrates multi-threading and inter-process communication (IPC) in C#. The goals are to create a multi-threaded application using true parallel threads, apply synchronization mechanisms such as mutexes, and implement IPC using named pipes.

\section{Implementation Details}

\subsection{Multi-Threading Implementation}
The project follows a four-phase threading approach:
\begin{itemize}
\item \textbf{Phase 1: Basic Multi-Threading} - Creates multiple threads to execute concurrent operations.
\item \textbf{Phase 2: Synchronization with Mutex} - Implements mutexes to prevent race conditions.
\item \textbf{Phase 3: Deadlock Demonstration} - Simulates a deadlock scenario using multiple mutexes.
\item \textbf{Phase 4: Deadlock Resolution} - Implements a proper resource acquisition strategy to prevent deadlocks.
\end{itemize}

\subsection{IPC Implementation}
IPC is implemented using Named Pipes to allow communication between separate processes:
\begin{itemize}
\item A \textbf{Pipe Server} listens for connections and sends messages.
\item A \textbf{Pipe Client} connects to the server and receives messages.
\item The server is extended to handle multiple clients concurrently.
\end{itemize}

\section{Challenges and Solutions}
\begin{itemize}
\item \textbf{Thread Synchronization} - Using mutexes prevented race conditions but required careful resource management.
\item \textbf{Deadlocks} - Implementing proper locking sequences avoided circular wait conditions.
\item \textbf{IPC Handling} - Named Pipes needed proper connection handling for multiple clients.
\end{itemize}

\section{Results and Testing}
\subsection{Threading Testing}
\begin{itemize}
\item Verified thread execution using console logs.
\item Tested mutex locks by running simultaneous operations on shared resources.
\item Simulated deadlock and confirmed process freezing.
\item Applied deadlock resolution and ensured smooth execution.
\end{itemize}

\subsection{IPC Testing}
\begin{itemize}
\item Verified client-server communication using Named Pipes.
\item Ran multiple clients to test concurrent connections.
\item Ensured proper data transmission and error handling.
\end{itemize}

\section{Conclusion}
This project enhanced understanding of multi-threading and IPC in C#. Implementing synchronization mechanisms, handling deadlocks, and using Named Pipes for process communication provided valuable experience in concurrent programming.

\end{document}
